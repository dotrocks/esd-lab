\chapter{Conclusion}

In this laboratory work, we explored the fundamental characteristics and behavior of Operational Amplifiers (Op-Amps) through theoretical analysis, LTspice simulations, and practical experiments. Our primary objectives were to understand the voltage gain of Op-Amp circuits, observe the input and output voltage waveforms, and analyze the clipping phenomenon.

\section{Key Findings}

\subsection{Theoretical Analysis}

The theoretical calculations for the voltage gain of an inverting and non-inverting amplifier were verified. We found that the expected voltage gain matched the formula \( A_v = 1 + \frac{R2}{R1} \) for the non-inverting configuration and \( A_v = -\frac{R2}{R1} \) for the inverting configuration. For our circuit with \( R1 = 100\Omega \) and \( R2 = 2.2k\Omega \), the calculated gain was 23 for the non-inverting amplifier and -22 for the inverting amplifier.

\subsection{LTspice Simulations}
The LTspice simulations successfully validated our theoretical predictions. The simulated output voltage closely aligned with the calculated values, demonstrating the accuracy of our theoretical analysis. The input and output waveforms were appropriately labeled and matched the expected sinusoidal shapes.

\subsection{Laboratory Exercises}
In the practical implementation, we observed the input and output voltages using an oscilloscope. The experimental results were consistent with both the theoretical and simulated data, confirming the expected voltage gain.

We identified the point of clipping by gradually increasing the input voltage amplitude. The maximum input voltage before clipping occurred was recorded, and the output waveform showed the expected distortion once the Op-Amp reached its supply voltage limits.

\section{Observations \& Analysis}

\subsection{Clipping Phenomenon}
Clipping occurs when the output voltage of the Op-Amp reaches its maximum or minimum limits, resulting in a flat-topped waveform instead of a smooth sinusoidal shape. This happens because the Op-Amp cannot produce an output voltage beyond its supply voltage.

\subsection{Measurement Accuracy}
Our measurements were accurate and closely matched the theoretical and simulated results. Minor discrepancies could be attributed to practical limitations such as component tolerances and measurement errors.

\section{Learning Outcomes}

This lab exercise enhanced our understanding of Op-Amps, specifically how they amplify signals and the conditions under which they operate linearly versus non-linearly (clipping).

We gained practical skills in building and testing electronic circuits, using LTspice for circuit simulations, and accurately measuring and interpreting waveform data with an oscilloscope.

In summary, the laboratory work provided a comprehensive insight into the operational principles of Op-Amps. It highlighted the importance of combining theoretical knowledge with practical skills to analyze and interpret circuit behavior effectively. The successful completion of this lab reinforces the foundational concepts essential for advanced electronic system design.
